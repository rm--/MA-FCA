\newglossaryentry{Algebraischer Datentyp}{
    name=Algebraischer Datentyp,
    description={ist ein Haskell-Datentyp, der die Kombination von Typen mittels algebraischer Operationen Summe und Produkt erlaubt (siehe Kapitel \ref{chapter:haskell})}
}

\newglossaryentry{Begriffsverband}{
    name=Begriffsverband,
    description={ist ein Verband in dem sich alle Formalen Begiffe eines Formalen Kontextes befinden (Kapitel \ref{subchapter:concept-lattice})}
}

\newglossaryentry{Benchmark}{
    name=Benchmark,
    description={bezeichnet ein Problem der \gls{Termination Problems Data Base}}
}

\newglossaryentry{Competition}{
    name=Competition,
    description={ist die Bezeichnung für einen \gls{Job}, der an der jählichen \gls{Termination-Competition} teilnimmt}
}

\newglossaryentry{DOT}{
    name=DOT,
    description={ist eine Sprache zur Beschreibung von Graphen für das Graphviz-Programm dot\footnote{\url{http://www.graphviz.org/Documentation/dotguide.pdf}}}
}

\newglossaryentry{DSL}{
    name=DSL,
    description={(Domain-Specific Language) eigens für eine spezielle Aufgabe geschriebene Programmiersprache}
}

\newglossaryentry{Formale Begriffsanalyse}{
    name=Formale Begriffsanalyse,
    description={beschreibt eine Methode des Data-Minings (Kapitel \ref{chapter:fca})}
}

\newglossaryentry{Formaler Begriff}{
    name=Formaler Begriff,
    description={ist ein grundlegender Begriff der \glslink{Formale Begriffsanalyse}{Formalen Begriffsanalyse} (Kapitel \ref{subchapter:fca-concept})}
}

\newglossaryentry{Formaler Kontext}{
    name=Formaler Kontext,
    description={ist ein grundlegender Begriff der \glslink{Formale Begriffsanalyse}{Formalen Begriffsanalyse} (Kapitel \ref{subchapter:fca-context})}
}

\newglossaryentry{GET}{
    name=GET,
    description={HTTP-Methode zum Lesen von Web-Inhalten}
}

\newglossaryentry{GHC}{
    name=GHC,
    description={(Glasgow Haskell Compiler) ist ein Compiler für Haskell\footnote{\url{https://www.haskell.org/ghc/}}}
}

\newglossaryentry{Handler}{
    name=Handler,
    description={beschreibt eine Funktion, die im \glslink{Yesod Web Framework} die Aktionen bezüglicher einer \gls{Route} behandelt}
}

\newglossaryentry{Haskell}{
    name=Haskell,
    description={funktionale Programmiersprache (Kapitel \ref{chapter:haskell})}
}

\newglossaryentry{Hasse-Diagramm}{
    name=Hasse-Diagramm,
    description={eine kompakte Darstellungform für Ordnungen (Kapitel \ref{subchapter:math-theory})}
}

\newglossaryentry{HTML}{
    name=HTML,
    description={(HyperText Markup Language) Beschreibungssprache für Web-Stukturen\footnote{\url{https://developer.mozilla.org/en-US/docs/Glossary/HTML}}}
}

\newglossaryentry{Job}{
    name=Job,
    description={beschreibt auf StarExec eine Kombination aus einer Menge Benchmarks und einer Menge Solver}
}

\newglossaryentry{JobPair}{
    name=JobPair,
    description={bezeichnet auf StarExec das Resultat eines Solvers bezüglich eines Benchmarks}
}

\newglossaryentry{jQuery}{
    name=jQuery,
    description={ist eine schnelle, kleine Javascript-Bibliothek die viele Funktionen unter anderem zum Traversieren und Manipulieren von HTML-Dokumenten besitzt. \footnote{\url{https://api.jquery.com/}}}
}

\newglossaryentry{LOC}{
    name=LOC,
    description={(Lines Of Code) engl. für Anzahl der Zeilen eines Quelltextes}
}

\newglossaryentry{Record-Typ}{
    name=Record-Typ,
    description={ist ein Haskell-Datentyp mit benannten Feldern (siehe Kapitel \ref{chapter:haskell})}
}

\newglossaryentry{SMT}{
    name=SMT,
    description={(Satisfiability Modulo Theories) \enquote{überprüft die Erfüllbarkeit von logischen Formeln bezüglich einer oder mehrerer Theorien}\footnote{\url{http://research.microsoft.com/en-us/um/people/leonardo/sbmf09.pdf}}}
}

\newglossaryentry{Solver}{
    name=Solver,
    description={bezeichnet ein Problem lösendes Computerprogramm (Kapitel \ref{subchapter:solver})}
}

\newglossaryentry{SQL}{
    name=SQL,
    description={(Structured Query Language) ist eine Datenbank-Abfragesprache}
}

\newglossaryentry{Star-Exec-Presenter}{
    name=Star-Exec-Presenter,
    description={ist eine in dem Haskell Web-Framework geschriebene Web-Applikation zur Visualisierung und Analyse von \gls{StarExec} (Kapitel \ref{subchapter:sep})}
}

\newglossaryentry{StarExec}{
    name=StarExec,
    description={ein Service der University of Iowa und der University of Miami (Kapitel \ref{subchapter:se})}
}

\newglossaryentry{SVG}{
    name=SVG,
    description={(Scalable Vector Graphics), engl. für skalierbare Vektorgrafik}
}

\newglossaryentry{Termination}{
    name=Termination,
    description={beschreibt das (An-)Halten eines Programmes auf eine Eingabe nach einer endlichen Anzahl von Arbeitsschritten}
}

\newglossaryentry{Termination Problems Data Base}{
    name=Termination Problems Data Base,
    description={ist eine Sammlung von Problemen die während der \gls{Termination-Competition} gelöst werden (Kapitel \ref{subchapter:tpdb})}
}

\newglossaryentry{Termination-Competition}{
    name=Termination-Competition,
    description={jährlich stattfindender Wettbewerk (Kapitel \ref{chapter:termcomp})}
}

\newglossaryentry{TPDB}{
    name=TPDB,
    description={siehe \gls{Termination Problems Data Base}}
}

\newglossaryentry{TRS}{
    name=TRS,
    description={(Term Rewriting System) engl. für Termersetzungssystem}
}

\newglossaryentry{Query}{
    name=Query,
    description={Query engl. für eine Abfrage, die an ein System gestellt werden kann z.b. Datenbank via \gls{SQL}}
}

\newglossaryentry{URL}{
    name=URL,
    description={(Uniform Resource Locator) engl. für einheitlicher Ressourcenzeiger ist die eindeutige Bezeichnung einer Resource in einem Netzwerk (häufige Anwendung im Web)\footnote{\url{https://tools.ietf.org/html/rfc3986\#section-1.1.3}}}
}

\newglossaryentry{Verband}{
    name=Verband,
    description={Verband}
}

\newglossaryentry{XML}{
    name=XML,
    description={(eXtensible Markup Language) ist eine generische vom W3C (World Wide Web Consortium)\footnote{\url{https://www.w3.org/}} spezifizierte Sprache}
}

\newglossaryentry{Yesod Web Framework}{
    name=Yesod Web Framework,
    description={ist ein in \gls{Haskell} von Michael Snoyman inital geschriebenes Web-Framework (Kapitel \ref{subchapter:yesod})}
}
