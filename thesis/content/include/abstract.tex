\chapter*{Abstract}

\pagenumbering{Roman}
\setcounter{page}{1}

Bei der Termination-Competition treten Computer-Programme sogenannte Solver gegeneinander an, um die Termination, das Halten eines Programms in verschiedenen Berechnungsmodellen automatisch zu beweisen. Die Solver lösen in einem Wettstreit eine Vielzahl von Problemen, die Benchmarks.

Die StarExec-Plattform stellt Kapazitäten für die Berechnungen und eine rudimentäre Auflistung der Ergebnisse. Stefan von der Krone und Prof. Waldmann implementierten Star-Exec-Presenter um eine bessere und flexiblere Visualisierungsschicht für die StarExec-Plattform zu haben.

In dieser Arbeit soll Formale Begriffsanalyse genutzt werden um Visualisierung und Analyse in Star-Exec-Presenter anzureichern. Die Formale Begriffsanalyse ist eine Methode des Data-Minings und erlaubt es Resultatdaten der Wettbewerbe zu klassifizieren und besser auswerten zu können.

\vspace{.5cm}
\begin{center}
\rule{0.5\textwidth}{.4pt}
\end{center}
\vspace{.5cm}
During the termination competition computer programs battle each other to automatically proove the termination in different models of computation. These computer programs, also known as solvers solve a lot of problems called benchmarks.

The StarExec-Platform provides the computation capacities and implements a primitive listing of the results. To achieve a better and more flexible visualization layer, Star-Exec-Presenter was developed by Stefan van der Krone and Prof. Waldmann.

The aim of this thesis is to enrich the visualization and analysis data from Star-Exec-Presenter by means of formal concept analysis. This technique, which is common in data mining, is used to classify the results of the termination competition in order to analyse them in more depth.
